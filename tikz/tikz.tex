
\section{Tikz} % (fold)
\label{sec:tikz}

\begin{frame}{O que é?}
\begin{block}{}
	\begin{itemize}
		\item Um dos melhores pacotes para {\bf produzir} gráficos \underline{vetorizados} em \LaTeX.
		\item Possui um extensa \href{http://ftp.fau.de/ctan/graphics/pgf/base/doc/pgfmanual.pdf}{documentação}.
		\item Disponibiliza diversos \href{http://www.texample.net/tikz/examples/}{exemplos} de como utilizar o pacote.
	\end{itemize}
\end{block}
\end{frame}

\subsection*{Vantagens \& Desvantagens}
\begin{frame}{Vantagens}
	\begin{multicols}{2}		
	\input tikz/gnuplot
	\input tikz/polar
	\end{multicols}
\end{frame}

% section tikz (end)
\begin{frame}{Vantagens}
	\begin{multicols}{2}		
	\input tikz/square
	\input tikz/circuitikz
	\end{multicols}
\end{frame}

\begin{frame}{Vantagens}	
	\input tikz/neural
\end{frame}


\begin{frame}{Desvantagens}
	\begin{block}{}
		\begin{itemize}
			\item Péssima curva de aprendizado.
			\item Desalinhamento de  versões entre o disponibilizado nos repositórios com os apresentados nos exemplos.
			\item {\it Sujo}
		\end{itemize}
	\end{block}
\end{frame}

\subsection*{Exemplos}
\begin{frame}{Circuito}
	\lstinputlisting[linewidth=8cm,basicstyle=\tiny\ttfamily]{tikz/circuitikz.tex}
\end{frame}

\begin{frame}{GnuPlot}
	\lstinputlisting[linewidth=8cm,basicstyle=\tiny\ttfamily]{tikz/gnuplot.tex}
\end{frame}

\begin{frame}[shrink=30]{Neural Network}
	\lstinputlisting[linewidth=8cm,basicstyle=\tiny\ttfamily]{tikz/neural.tex}
\end{frame}