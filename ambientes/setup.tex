\section{Configuração de ambiente} % (fold)
\label{sec:configura_o_de_ambiente}

\subsection{Sistemas base Unix} % (fold)
\label{sub:sistemas_base_unix}

\begin{frame}[fragile]
	\frametitle{Unix}

	Para sistemas Unix, os ambientes podem ser instalados diretamente via terminal, com o respectivo comando:

	\begin{description}
		\item [Fedora] {\code \# dnf install texlive-* texmaker ibus-qt}
		\item [Ubuntu] {\code \# apt install texlive-full texmaker ibus-qt}
		\item [Mac-OS] {\code \# brew install texlive-full texmaker ibus-qt}
	\end{description}

\end{frame}

\subsection{Sistemas base Windows} % (fold)
\begin{frame}[fragile]
	\frametitle{Windows}

	Para sistemas {\it Windows} os executáveis podem ser localizados nos seguintes {\it links}:

	\begin{description}
		\item [MiKTeX] \href{http://ctan.sharelatex.com/tex-archive/systems/win32/miktex/}{\beamergotobutton{Link} }
		\item [TexMaker] \href{http://www.xm1math.net/texmaker/download.html#windows}{\beamergotobutton{Link} }
	\end{description}

\end{frame}

